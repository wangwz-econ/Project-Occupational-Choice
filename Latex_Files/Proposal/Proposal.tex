\documentclass[12pt]{article}

% DEFAULT PACKAGE SETUP
% DEFAULT PACKAGE SETUP
\usepackage{setspace,graphicx,epstopdf,amsmath,amsfonts,amssymb,amsthm,geometry,fancyvrb}
\usepackage{marginnote,datetime,enumitem,rotating}
\usepackage{threeparttable,booktabs,float,soul}
\usdate
\geometry{scale=0.8}
\usepackage{titlesec}
\titleformat{\paragraph}[runin]{\itshape}{}{}{}[.]
\titlelabel{\thetitle.\;}
\usepackage{indentfirst}
\setlength{\parindent}{0pt}
\setlength{\parskip}{8pt}
\usepackage{fourier}  

%% Use natbib.sty.
\usepackage{natbib,fancybox,url,graphicx,color}
\definecolor{MyBlue}{rgb}{0,0.2,0.6}
\definecolor{MyRed}{rgb}{0.4,0,0.1}
\definecolor{MyGreen}{rgb}{0,0.4,0}
\newcommand{\highlightR}[1]{{\color{MyRed}{#1}}} 
\newcommand{\highlightB}[1]{{\color{MyBlue}{#1}}} 
\usepackage[bookmarks=true,bookmarksnumbered=true,colorlinks=true,linkcolor=MyGreen,citecolor=MyBlue,filecolor=MyBlue,urlcolor=MyGreen]{hyperref}
\bibliographystyle{econ}

%% Theorem Environment
\theoremstyle{definition}
\newtheorem{assumption}{Assumption}
\newtheorem{definition}{Definition}
\newtheorem{theorem}{Theorem}
\newtheorem{proposition}{Proposition}
\newtheorem{lemma}[theorem]{Lemma}
\newtheorem{example}[theorem]{Example}
\newtheorem{remark}[theorem]{Remark}
\usepackage{math}


\begin{document}
	%%%%%%%%%%%%%%%%%%%%%%%%%%%%%%%%%%%%%%%%%%%%%%%%%%%%%%%%%%
	%%%%%%%      Title, Author and Abstract
	%%%%%%%%%%%%%%%%%%%%%%%%%%%%%%%%%%%%%%%%%%%%%%%%%%%%%%%%%%
\title{\bf {Sector Choices in Incomplete Markets: The Role of Idiosyncratic Production Risk and Financial Imperfections}} 
\author{Wenzhi Wang} 
\date{\today}
\maketitle

\begin{spacing}{1}

{\noindent \bf Goal}

This paper investigates how uninsurable labor earnings risk affects individuals' education and sector choices. Unlike other incomplete market models, in this paper, the idiosyncratic earnings risk comes from the sector-specific production risk, which transmits differentially across sectors due to financial imperfections. In our context, this means that only state-owned enterprises (SOE) can get access to the credit market, while privately-owned enterprises (POE) are subject to natural borrowing limit and can only use the self-financing channel to counter production risks.

{\noindent \bf Problems Remain to Be Solved}

What is the proper way to aggregate the household side state variables to obtain the variables in the production side?

How to use the idiosyncratic production risk and credit market imperfections to pin down sector specific labor earnings risk? Since it is a macro model, it seems necessary to introduce some equilibrium (and/or invariant distribution) concepts. To do that, some explicit aggregation rules in the household side are unavoidable.

To model the production side, since there is heterogeneity in firms, do I need to distinguish producers from workers who make sector choices? Or should I just let the workers employed in the private sector essentially be self-employed and run business and directly be affected by production risks.

{\noindent \bf Related Literature}

Papers that investigate how uninsurable risks affect people's job (or occupation or industry) mobility: \citet{cubas2020, kambourov2009, dillon2018, neumuller2015, hawkins2016, cubas2023, cubas2017, low2010, liu2019, altonji2013, dvorkin2019, lee2006}.

Papers that study individuals' human accumulation choices in incomplete markets: \citet{singh2010, krebs2003, park2018, huggett2011, stantcheva2017, huggett2006}.

Papers that study the effects of financial imperfections and production side literature: \citet{buera2011, angeletos2006, krusell2000, midrigan2014, song2011}.

Papers that are most relevant in terms of modelling ingredients:
\begin{itemize} 
	\setlength\itemsep{0em}
	\setlength{\parskip}{8pt} 
	\setlength{\topsep}{0em} 
	\item Take the general equilibrium framework in \citet{heckman1998} but add sector choices and uninsurable risks.
	\item Use the on-the-job human capital accumulation process (and most notations) in \citet{huggett2011} but explicitly model schooling choice, according to which people are categorized into different skill groups.
	\item Use the idea of the human capital transferability matrix in \citet{dvorkin2019}, the idea of how production side risk translates into people's wage risk in \citet{kambourov2009} \footnote{\highlightR{NOT SO SURE!}}, and the idea of financial imperfections in \citet{song2011}\footnote{\highlightR{NOT SO SURE!}}.
\end{itemize}

%%%%%%%%%%%%%%%%%%%%%%%%%%%%%%%%%%%%%%%%%%%%%%%%%%%%%%%%%%
%%%%%%%     Section 1: HH Side
%%%%%%%%%%%%%%%%%%%%%%%%%%%%%%%%%%%%%%%%%%%%%%%%%%%%%%%%%%

\section{Individuals' Optimization Problem} \label{sec:hh}

\highlightR{Overview of the Setup}:
Individuals take all the factor prices as given. In the first period (when the agent turns 18), he (as a high school graduate) needs to first choose whether to attend college, according to which they are classified as high-skilled or low-skilled groups. Then in each working period, the worker draws a transitory labor income shock, then chooses which sector he wants to work in, how much percentage of time he allocates to accumulate on-the-job human capital and how much income is he wants to consume. During his working period, the accumulation of on-the-job human capital follows a (deterministic) skill-specific process. Human capital is of a general form, meaning that it can be carried over to another sector (though with some switch cost) when he decides to change his sector. After retirement, he depends on his savings for a living. All individuals are subject to the natural borrowing limit, the same concept as in \citet{aiyagari1994}. 

\subsection{Working Periods: Decisions on Sector and Human Capital Accumulation}

At date $t$, individuals of cohort $c$ become age $\undl{a}=18$. At this age, individuals differ from each other only in an endowment vector $\bds{x}_t = \left(h_{\undl{a}t}, k_{\undl{a}t}, \theta\right)$, where $h_{\undl{a}t}$, $k_{\undl{a}t}$ denote his initial human capital stock and wealth level, respectively, and $\theta$ is his learning ability drawn from a distribution $G(\theta)$. He needs to make a choice on whether to go to college based on different discounted lifetime utility. Attending college makes an individual a high-skilled worker (labeled as $H$) while engaging in the labor market directly means that he is a low-skilled worker (labeled as $L$). 

For those workers in the labor market (from age $\undl{a}^L = 18$ for $L$-type workers and age $\undl{a}^H = 22$ for $H$-type workers to the retirement age $a^{r} = 60$), in each period, he needs to choose between two different occupations or sectors, $j \in \{S, P\}$--whether to work in state-owned enterprises (labeled as $S$) or privately owned companies (labeled as $P$). \highlightR{From individuals' point of view, sectors differ from each other in the aspect of mean earnings and the idiosyncratic earning risks} (which are originated from sector's idiosyncratic production risks and potentially amplified by financial imperfection in the credit market) and mean earnings (which are pinned down by the general equilibrium). 

After retirement at the age of $a^r=60$, the individual depends only on his savings and live up to $\ol{a} = 70$ years old.

Let's first talk about people's sector decisions and human capital decisions after choosing whether to attend college. In period $t$, each individual aged $a$ who is currently in skill group $\tau \in \bc{L, H}$ and in sector $j_{at} \in \bc{S, P}$ faces the following problem:
\begin{equation} \label{uncond value function}
	V_{at} \of{\tau, j_{at}, h_{at}, k_{at}, \theta} = \max \bc{ W_{at}^S\of{\tau, j_{at}, h_{at}, k_{at}, \theta}, W_{at}^P\of{\tau, j_{at}, h_{at}, k_{at}, \theta}},
\end{equation}
where the choice-specific value function $W_{at}^S$, and $W_{at}^P$ are defined by
\begin{equation} \label{choice-specific vf}
	W_{at}^j\of{\tau, j_{at}, h_{at}, k_{at}, \theta} = \max_{c_{at}, l_{at}, j_{a+1,t+1}} \bc{u(c_{at}) + \beta EV_{a+1,t+1} \of{\tau, j_{a+1, t+1}, h_{a+1,t+1}, k_{a+1,t+1}, \theta}}, 
\end{equation}
subject to,
\begin{equation} \label{consumption}
	c_{at} + k_{a+1,t+1} = k_{at}\bp{1+r_{t+a-1}} + e_{at} - T_{a, t+a-1}, \; a=\undl{a}^\tau, \ldots, \ol{a} \text { and } k_{\ol{a}+1}=0 ;
\end{equation}
\begin{equation}
	e_{at} = w_{t+a-1}^j h_{at} l_{at} \text{ if } a < a^r, \text{ and } e_a = 0, \text{ otherwise};
\end{equation}
\begin{equation} 
	h_{a+1} = \footnote{\highlightR{I am still unsure how to model the human capital accumulation process. I can actually take a more general specification as in \citet{heckman1998} (for the non-switchers): $$h_{a+1, t+1}=A^\tau(\theta)\left(1-l_{a, t}\right)^{\alpha_\tau}\left(h_{at}\right)^{\beta_\tau}+\left(1-\sigma^\tau\right) h_{a, t}^S,$$ which incorporates human capital depreciation and can better differentiate the effects of schooling $\tau$ and ability $\t$. Maybe, in solving and estimating the model, I should try this more sophisticated specification as well.}}
	\begin{cases}
		h_{at} + \t \bs{h_{at} \bp{1-l_{at}}}^{\a_\tau} & \text{ if } j_{a+1, t+1} = j_{at}, \\
		\d_{j_{at}j_{a+1, t+1}} h_{at} + \t \bs{\d_{j_{at}j_{a+1, t+1}} h_{at} \bp{1-l_{at}}}^{\a_\tau} & \text{ if } j_{a+1, t+1} \neq j_{at}.
	\end{cases}	 
\end{equation}

Notations:
\begin{itemize}
	\setlength\itemsep{0em}
	\setlength{\parskip}{8pt} 
	\setlength{\topsep}{0em} 
	\item $r_{t+a-1}$: the real return of capital;
	\item $T_{a, t+a-1}$: taxes (\highlightR{not so sure how to deal with it});
	\item $l_{at}$: the choice vaiable indicating how much time is allocated to work, while all the other time is allocated to accumulate on-th-job human capital (total time is normalized to 1);
	\item $e_{at}$: labor earnings;
	\item $w_{t+a-1}^j$: sector-specific wage rate (\highlightR{The only source of stochasticity in our model. Come from the production side! Still not so sure how to model it. But for now, we can separately model the partial equilibrium model, with $w_{t}^j = \rho w_{t-1}^j + \ve_{t}^j$, where $\ve_{t}^j$ is a normal distributed random variables. Its variance implies the sector-specific labor earning risks.})
	\item $\d_{j_{at}j_{a+1, t+1}}$: human capital transferability parameter (two parameters, $\d_{SP}, \d_{PS}$) indicating the remaining human capital if the individual switches from sector $j_{at}$ to sector $j_{a+1, t+1}$;
	\item $\t$: ability, drawn from $G\of{\t}$ (can be estimated in the last step using SMM to match some key moments, as in \citet{huggett2011}); 
	\item $\a_\tau$: schooling-specific human capital accumulation parameter.
\end{itemize}

Several remarks:
\begin{itemize}
	\setlength\itemsep{0em}
	\setlength{\parskip}{8pt} 
	\setlength{\topsep}{0em} 
	\item The two human capital transferability parameters fully characterize the sector switching cost. Here, human capital is not sector-specific becasue occupations and industries can overlap between SOEs and POEs.
	\item The most important formualtion is the deterministic huamn capital process. Maybe, when I later solve the model, I can first try some easy specification as in \citet{dvorkin2019}. In their paper, the human capital evolution process can be fully characterized by a human capital transferability matrix, and stochasticity is brought in by multiplying occupation-specific labor market opportunities shocks.
	\item As in \citet{heckman1998}, $h$ and $A^\tau(\theta)$ represent ability to ``earn'' and ability to ``learn'', respectively, measured after completion of school. They embody the contribution of schooling to subsequent learning and earning in the schooling-level $\tau$-specific skills as well as any initial endowments.
\end{itemize}

\subsection{Schooling Choice}

Assum that in period $t$, an individual turns $\undl{a}=18$ years old. Then he needs to choose whether to attend college by comparing the following value functions:
\begin{equation}
	\tau = \argmax \Bigl\{V\of{L, j_{\undl{a}t}, h_{{\undl{a}t}}, k_{\undl{a}t}, \t},\; V\of{H, j_{\undl{a}+4, t+4}, h_{{\undl{a}t}}, k_{\undl{a}t}, \t} - D^H\Bigl\} ,
\end{equation}
where $D^H$ is the discounted direct cost of schooling (It can also be incorporated in to the value function, i.e., $V\of{H, j_{\undl{a}+4, t+4}, h_{{\undl{a}t}}, k_{\undl{a}t}-D^H, \t}$). 

\subsection{Aggregation}





\newpage
\bibliography{../Reference/reference.bib}

\end{spacing}
\end{document}
